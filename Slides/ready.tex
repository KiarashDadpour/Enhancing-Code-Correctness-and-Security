\documentclass[t,ignorenonframetext]{beamer}
\usepackage{beamerinnerthemecircles}
\usepackage{tcolorbox}
\mode<presentation>
{
\usetheme{Warsaw}
\usecolortheme{whale}
\setbeamercolor{frametitle}{fg=black,bg=lightgray}
\usefonttheme{structurebold}
}

\title[Security & AI Overview]{Enhancing Code Correctness and Security with Large Language Models}
\author[University of Guilan]{Dr. Reza Ebrahimi Atani \newline Dr. Amir Tabatabaei \newline Asal Mahmodi Nezhad \newline Kiarash Dadpour}
\date{Fall, 2025}

\begin{document}

\frame{\maketitle}

\section{Introduction}
\begin{frame}{Introduction: Computer Security and AI Overview}
\begin{itemize}
\item The growing importance of computer security in the digital era.
\item Advances in artificial intelligence (AI) are transforming technology and security.
\item Understanding theoretical foundations is key for research and application.
\end{itemize}
\end{frame}

\begin{frame}{Introduction: Basic Concepts}
\begin{itemize}
\item Computer Security: protecting systems from unauthorized access and attacks.
\item Network Security: securing communication protocols and data movement.
\item Artificial Intelligence: enabling machines to simulate human-like intelligence.
\end{itemize}
\end{frame}

\begin{frame}{Introduction: Importance in Academia and Industry}
\begin{itemize}
\item Universities focus on foundational theory and applied security research.
\item Industries utilize AI and security protocols to protect data assets.
\item Bridging theory and practice is essential for innovation in both fields.
\end{itemize}
\end{frame}

\section{Large Language Models (LLM)}
\begin{frame}{Large Language Models: Definition and Purpose}
\begin{itemize}
\item LLMs are AI models trained on vast amounts of text data.
\item Purpose: to understand and generate human-like language.
\item Examples: GPT, BERT, and other transformer-based architectures.
\end{itemize}
\end{frame}

\begin{frame}{Large Language Models: Architecture}
\begin{itemize}
\item Based on Transformer models featuring self-attention mechanisms.
\item Layers of multi-head attention and feed-forward networks.
\item Trained using unsupervised or semi-supervised learning on massive text corpora.
\end{itemize}
\end{frame}

\begin{frame}{Large Language Models: Applications and Challenges}
\begin{itemize}
\item Applications: natural language processing, code generation, chatbots.
\item Security concerns: data privacy, adversarial attacks, bias in models.
\item Continual research to improve robustness and ethical use of LLMs.
\end{itemize}
\end{frame}
\section{Prompt Engineering Patterns}
\begin{frame}{Patterns}
\begin{itemize}
\item Persona Pattern
\item Audience Persona Pattern
\item Few Shots \& Zero Shot Example
\end{itemize}
\end{frame} 

\section{Attacks}
\begin{frame}{Buffer Overflow: Abstract}
\begin{itemize}
    \item A buffer overflow occurs when data larger than the buffer’s capacity is written into memory, causing overwriting of adjacent memory areas.
    \item This vulnerability can allow execution of malicious code and remains a serious threat even today.
    \item Languages like C and C++ are most vulnerable, since they lack automatic bounds checking and rely on manual memory management.
    \item Python significantly reduces the risk of buffer overflow through automatic memory management and strict bounds checking.
\end{itemize}
\end{frame}


\begin{frame}{Buffer Overflow: Example (Part 1)}
\begin{figure}[htb]
	\centering
	\includegraphics[width=0.8\linewidth]{5.1.png}
\end{figure}
\end{frame}
\begin{frame}{Buffer Overflow: Example (Part 2)}
\begin{figure}[htb]
	\centering
	
 \includegraphics[width=0.9\linewidth]{5.3.png}
\end{figure}
\end{frame}

\begin{frame}{Buffer Overflow: Example (Part 3)}
\begin{figure}[htb]
	\centering
	
 \includegraphics[width=0.8\linewidth]{5.2.png}
\end{figure}
\end{frame}

\end{frame}
\begin{frame}{Buffer Overflow: Example (Part 4)}
\begin{figure}[htb]
	\centering
	\includegraphics[width=0.9\linewidth]{5.4.png}
\end{figure}

\end{frame}
\begin{frame}{Buffer Overflow: Prompt}
\begin{tcolorbox}
[colback=blue!5!white,colframe=navy!75!black,title=Persona]
You are a senior application security engineer and memory-safety specialist with experience in identifying,
assessing, and mitigating buffer overflow bugs and memory-corruption vulnerabilities in native code 
(C/C++/Objective‑C and native extensions).Be technical, precise, and pragmatic.
When needed, reference OWASP secure coding and testing guidelines.
\end{tcolorbox}
\end{frame}
\begin{frame}{Buffer Overflow: Prompt}
\begin{tcolorbox}
[colback=blue!5!white,colframe=navy!75!black,title=Context]
You receive code that may contain potentially dangerous inputs (e.g., network input, files, or command-line arguments).
Even if modern OS protections like ASLR and NX are enabled, the code must be secured.
You can use sanitizers to help detect issues.
\end{tcolorbox}
\end{frame}

\begin{frame}{Buffer Overflow: Prompt}
\begin{tcolorbox}
[colback=blue!5!white,colframe=navy!75!black,title=Tasks (Part 1)]
\begin{itemize}
    \item  Perform a statistical review of files and note the exact location (file + line range) of suspicious inputs (max 8 lines as evidence).
    \item Classify each issue (stack / heap / off‑by‑one / format / integer overflow).
    \item Determine severity (LOW / MEDIUM / HIGH / CRITICAL) with explanation.
\end{itemize}
\end{tcolorbox}
\end{frame}

\begin{frame}{Buffer Overflow: Prompt}
\begin{tcolorbox}
[colback=blue!5!white,colframe=navy!75!black,title=Tasks (Part 2)]
\begin{itemize}

\item Add a risk table or matrix: likelihood  impact exploitability.
\item Check the code’s compliance with OWASP guidelines and secure coding standards.
\item Prioritize tasks: review the most dangerous inputs and buffers first.
\end{itemize}
\end{tcolorbox}
\end{frame}

\begin{frame}{Insecure Deserialization: Abstract}
\begin{itemize}
    \item Insecure deserialization occurs when serialized data is deserialized back into objects without validating the source or content.
    \item A tampered serialized payload can let an attacker change program behavior, escalate privileges, or trigger arbitrary code execution.
    \item Automatic or direct object reconstruction from untrusted serialized input combined with lack of validation or integrity checks.
    \item Languages/platforms that support direct object deserialization (Java, PHP, and Python) are more susceptible.
\end{itemize}
\end{frame}

\begin{frame}{Insecure Deserialization: Example}
\begin{figure}[htb]
	\centering
	\includegraphics[width=0.9\linewidth]{6.1.png}
\end{figure}
\end{frame}
\begin{frame}{Insecure Deserialization: Prompt}
\begin{tcolorbox}
[colback=blue!5!white,colframe=navy!75!black,title=Persona]
You are a senior application security engineer specializing in insecure deserialization in
Java, Python, PHP, .NET, and custom binary formats. Be technical, precise, and pragmatic.
When references are needed, cite OWASP guidance. Never produce exploit code or step‑by‑step exploitation
instructions. 
\end{tcolorbox}
\end{frame}
\begin{frame}{Insecure Deserialization: Prompt}
\begin{tcolorbox}
[colback=blue!5!white,colframe=navy!75!black,title=Context]
You are reviewing source files, configuration, and build scripts for an application that 
deserializes external data (inputs may come from the network, files, cookies, queues, RPC, or plugins).
You may run safe unit tests and static analysis tools in an isolated environment.

\end{tcolorbox}
\end{frame}

\begin{frame}{Insecure Deserialization: Prompt}
\begin{tcolorbox}
[colback=blue!5!white,colframe=navy!75!black,title=Tasks (Part 1)]
\begin{itemize}
    \item  Find all locations where serialized data is deserialized (for example: unserialize, pickle.loads,
ObjectInputStream.readObject, BinaryFormatter.Deserialize, and framework bindings). Report the
file and line range for each finding (format: file: start_line–end_line). Provide up to 8 lines of 
evidence per finding. 
\item Label each finding as one of: - insecure deserialization - gadget risk - unsafe class resolution 
\end{itemize}
\end{tcolorbox}
\end{frame}
\begin{frame}{Insecure Deserialization: Prompt}
\begin{tcolorbox}
[colback=blue!5!white,colframe=navy!75!black,title=Tasks (Part 2)]
\begin{itemize}
    \item Write a short description of the problem for each finding. Assign a 
severity: LOW / MEDIUM / HIGH / CRITICAL. Include a brief justification for the chosen severity.
\item Produce a risk table for the set of findings. The matrix should include at minimum these
axes/columns: likelihood × impact × exploitability. Use this matrix to help prioritize remediation. 

\end{itemize}
\end{tcolorbox}
\end{frame}
\begin{frame}{Insecure Deserialization: Prompt}
\begin{tcolorbox}
[colback=blue!5!white,colframe=navy!75!black,title=Tasks (Part 3)]
\begin{itemize}
    \item Check how well the code aligns with OWASP guidance and secure‑coding standards. List deviations
from those standards.
\item Based on the risk matrix, write a short action plan that specifies which
locations (files/line ranges/modules) should be remediated first. 
\item Send me the corrected code for each part
\end{itemize}
\end{tcolorbox}
\end{frame}

\begin{frame}{SQL Injection: Abstract}
\begin{itemize}
    \item SQL Injection is one of the oldest and most dangerous software security vulnerabilities.
    \item It occurs when attackers inject malicious SQL code into user input fields.
    \item The attack exploits insecure dynamic SQL query construction without proper input validation or parameterization.
    \item A successful SQL Injection can lead to unauthorized access, modification, or deletion of data.
\end{itemize}
\end{frame}


\begin{frame}[fragile]{SQL Injection: Login}

\begin{columns}[T] % T برای تراز بالای ستون‌ها
    \begin{column}{0.48\linewidth} % ستون سمت چپ
        \includegraphics[width=\linewidth]{3.1.png}
    \end{column}
    
    \begin{column}{0.48\linewidth} % ستون سمت راست
        \texttt{%
        \newline
        \newline
        \newline
        \textcolor{red}{SELECT} * \\
        \textcolor{red}{FROM} users \\
        \textcolor{red}{WHERE} username = 'admin' \\
        \textcolor{red}{OR} '1'='1' \\
        \textcolor{red}{AND} password = 'anything'
        }
    \end{column}
\end{columns}

\end{frame}


\begin{frame}[fragile]{SQL Injection: Login}

\begin{columns}[T] % T برای تراز بالای ستون‌ها
    \begin{column}{0.48\linewidth} % ستون سمت چپ
        \includegraphics[width=\linewidth]{3.3.png}
    \end{column}
    
    \begin{column}{0.48\linewidth} % ستون سمت راست
        \texttt{%
        \newline
        \newline
        \newline
        \textcolor{red}{SELECT} * \\
        \textcolor{red}{FROM} users \\
        \textcolor{red}{WHERE} username = 'admin'; \\
        \textcolor{red}{DROP TABLE} users; \\
        \textcolor{red}{AND} password = 'anything'
        }
    \end{column}
\end{columns}

\end{frame}
\begin{frame}{SQL Injection: Example}
\begin{figure}[htb]
	\centering
	\includegraphics[width=1.1\linewidth]{3.4.png}
\end{figure}

\end{frame}

\begin{frame}{SQL Injection: Prompt}
\begin{tcolorbox}
[colback=blue!5!white,colframe=navy!75!black,title=Persona]
You are a senior application security engineer and database expert with hands-on experience finding, explaining, and remediating SQL Injection vulnerabilities across multiple languages and frameworks (PHP, Java, C#, Python, Node.js, Go, Ruby, etc.). Your tone is technical, precise and pragmatic. You reference OWASP best practices where relevant and always prefer safe, defensive guidance over exploit details.\
\end{tcolorbox}
\end{frame}

\begin{frame}{SQL Injection: Prompt}
\begin{tcolorbox}
[colback=blue!5!white,colframe=navy!75!black,title=Context]
You will be given one or more source files that interact with a database. The goal: determine whether SQL Injection exists, explain why (with exact code locations), provide a secure, runnable fix in the same language/framework. Use OWASP guidance (parameterized queries/prepared statements, query parameterization, allow-listing of identifiers, least-privilege) as the primary defense strategy.
\end{tcolorbox}
\end{frame}

\begin{frame}{SQL Injection: Prompt}
\begin{tcolorbox}[colback=blue!5!white,colframe=navy!75!black,title=Tasks (Part 1)]
\begin{enumerate}
  \item Analyze the code and identify any SQL Injection vulnerabilities. If none, explain why the code is safe.
  \item For each vulnerability found:
  \begin{itemize}
    \item Provide exact file/line references and an evidence snippet (<= 8 lines).
    \item Explain the technical root cause (e.g., dynamic string concatenation, unsafe ORM usage).
    \item Assign severity (LOW/MEDIUM/HIGH) with justification.
  \end{itemize}
  
\end{enumerate}
\end{tcolorbox}
\end{frame}

\begin{frame}{SQL Injection: Prompt}
\begin{tcolorbox}[colback=blue!5!white,colframe=navy!75!black,title=Tasks (Part 2)]
  \begin{enumerate}
    \setcounter{enumi}{2} % continue numbering

    \item Produce unit/integration tests that assert the fix prevents SQL injection (use DB mocks or in-memory DBs).

    \item NEVER output exploit payloads or instructions to exploit the vulnerability.
  \end{enumerate}
\end{tcolorbox}
\end{frame}


\begin{frame}{XSS: Abstract}
\begin{itemize}
  \item Malicious scripts are permanently saved on the target server (e.g., in databases, comment sections, forums).
    \item Executes every time a user loads the affected content, impacting many users simultaneously.
    \item Represents the highest risk due to persistence and scale.

\end{itemize}
\end{frame}
\begin{frame}{Types of XSS}
\begin{itemize}
    \item Stored-XSS: persistent server storage of malicious scripts
    \item Reflected-XSS: script comes from user request, reflected in server response
    \item DOM-based-XSS: client-side modification of the Document Object Model
\end{itemize}
\end{frame}
\begin{frame}{XSS: Prompt}
\begin{tcolorbox}
[colback=blue!5!white,colframe=navy!75!black,title=Persona]
You are a senior application security engineer and expert prompt-engineer with deep, hands-on experience finding, explaining,
triaging, and remediating Cross-Site Scripting (XSS) vulnerabilities across web stacks (server-side templating:
PHP/Twig, Python/Jinja2, Java/JSP, Ruby/ERB; client frameworks: React/Angular/Vue/vanilla JS). Your tone is technical, precise,
and pragmatic. Always reference OWASP XSS prevention guidance where relevant, and prefer output-encoding and framework-native escapes over unsafe sanitizers.


\end{tcolorbox}
\end{frame}

\begin{frame}{XSS: Prompt}
\begin{tcolorbox}
[colback=blue!5!white,colframe=navy!75!black,title=Context]
You will be given one or more source files (or a repository snapshot) and optionally runtime configuration \
(framework, templating engine, CSP, cookie flags). The code may include server‑rendered templates,
API endpoints that return HTML/JSON, frontend JS that manipulates DOM with user input, and third‑party widget integrations.
The environment may include client inputs from query params, request bodies, headers, cookies, WebSocket messages, 
or stored content (database/files). Assume you have read‑only access to the code and can run tests in an isolated environment if requested.


\end{tcolorbox}
\end{frame}

\begin{frame}{XSS: Prompt}
\begin{tcolorbox}
[colback=blue!5!white,colframe=navy!75!black,title=Tasks (Part 1)]
\begin{itemize}
    \item  Code & runtime config available: you have read‑only access to server + frontend source
    files and runtime config (templating engine, CSP, cookie flags); analyze based on these artifacts.
    \item Untrusted input sources to consider: query params, request bodies, headers, cookies, postMessage,
   WebSocket messages, and stored content (DB/files).
   
\end{itemize}
\end{frame}
\end{tcolorbox}
\end{frame}

\begin{frame}{XSS: Prompt}
\begin{tcolorbox}
[colback=blue!5!white,colframe=navy!75!black,title=Tasks (Part 2)]
\begin{itemize}
    \item Output contexts & sinks to prioritize: HTML body text, HTML attributes,
   JS string/identifier, URL context, CSS, HTML comments, and event handlers;
   flag dangerous sinks (innerHTML, document.write, eval, dangerouslySetInnerHTML, jQuery.html, direct DOM insertion).
   \item Verify framework protections: frameworks may auto‑escape—do not assume safety. 
   Check template settings and escape modes; if not provable, treat as untrusted.
    

\end{itemize}
\end{tcolorbox}
\end{frame} 

\begin{frame}{XSS: Prompt}
\begin{tcolorbox}
[colback=blue!5!white,colframe=navy!75!black,title=Tasks (Part 3)]
\begin{itemize}
\item Safe verification requirement: produce safe, non‑exploitative verification tests 
   (in‑memory DBs/mocks, headless assertions) and full remediation patches in the same language. Never output exploit payloads.
\end{itemize}
\end{tcolorbox}
\end{frame} 

\section{References}
\begin{frame}{References}
\tiny
\begin{itemize}
\item O. Iyama, K. Kato, J. Miyachi, \textit{Derived categories of $N$-complexes}, 2017.
\item A. Neeman, \textit{The homotopy category of flat modules and Grothendieck duality}, 2008.
\item OWASP Foundation, \textit{SQL Injection Prevention Cheat Sheet}.
\item Vaswani et al., \textit{Attention is All You Need}, 2017.
\item Radford et al., \textit{Language Models are Few-Shot Learners}, OpenAI, 2020.
\end{itemize}
\end{frame}

\begin{frame}
\vspace{3cm}
\begin{center}
\begin{LARGE}
\textcolor{blue}{\textbf{Thank you all for your attention}}
\end{LARGE}
\end{center}
\end{frame}

\end{document}
