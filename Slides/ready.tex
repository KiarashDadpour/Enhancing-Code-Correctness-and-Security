\documentclass[t,ignorenonframetext]{beamer}
\usepackage{beamerinnerthemecircles}
\usepackage{tcolorbox}
\mode<presentation>
{
\usetheme{Warsaw}
\usecolortheme{whale}
\setbeamercolor{frametitle}{fg=black,bg=lightgray}
\usefonttheme{structurebold}
}

\title[Security & AI Overview]{Enhancing Code Correctness and Security with Large Language Models}
\author[University of Guilan]{Dr. Reza Ebrahimi Atani \newline Dr. Amir Tabatabaei \newline Asal Mahmodi Nezhad \newline Kiarash Dadpour}
\date{Fall, 2025}

\begin{document}

\frame{
\maketitle
}

\section{Introduction}
\begin{frame}{Introduction: Computer Security and AI Overview}
\begin{itemize}
\item The growing importance of computer security in the digital era.
\item Advances in artificial intelligence (AI) are transforming technology and security.
\item Understanding theoretical foundations is key for research and application.
\end{itemize}
\end{frame}

\begin{frame}{Introduction: Basic Concepts}
\begin{itemize}
\item Computer Security: protecting systems from unauthorized access and attacks.
\item Network Security: securing communication protocols and data movement.
\item Artificial Intelligence: enabling machines to simulate human-like intelligence.
\end{itemize}
\end{frame}

\begin{frame}{Introduction: Importance in Academia and Industry}
\begin{itemize}
\item Universities focus on foundational theory and applied security research.
\item Industries utilize AI and security protocols to protect data assets.
\item Bridging theory and practice is essential for innovation in both fields.
\end{itemize}
\end{frame}

\section{Large Language Models (LLM)}
\begin{frame}{Large Language Models: Definition and Purpose}
\begin{itemize}
\item LLMs are AI models trained on vast amounts of text data.
\item Purpose: to understand and generate human-like language.
\item Examples: GPT, BERT, and other transformer-based architectures.
\end{itemize}
\end{frame}

\begin{frame}{Large Language Models: Architecture}
\begin{itemize}
\item Based on Transformer models featuring self-attention mechanisms.
\item Layers of multi-head attention and feed-forward networks.
\item Trained using unsupervised or semi-supervised learning on massive text corpora.
\end{itemize}
\end{frame}

\begin{frame}{Large Language Models: Applications and Challenges}
\begin{itemize}
\item Applications: natural language processing, code generation, chatbots.
\item Security concerns: data privacy, adversarial attacks, bias in models.
\item Continual research to improve robustness and ethical use of LLMs.
\end{itemize}
\end{frame}

\section{Attacks}
\begin{frame}{SQL Injection}
\textbf{Abstract}

\end{frame}

\begin{frame}{SQL Injection}
\begin{figure}[htb]
	\centering
	\includegraphics[width=1.1\linewidth]{3.4.png}
\end{figure}

\end{frame}



\begin{frame}[fragile]{SQL Injection}

\begin{columns}[T] % T برای تراز بالای ستون‌ها
    \begin{column}{0.48\linewidth} % ستون سمت چپ
        \includegraphics[width=\linewidth]{3.1.png}
    \end{column}
    
    \begin{column}{0.48\linewidth} % ستون سمت راست
        \texttt{%
        \newline
        \newline
        \newline
        \textcolor{red}{SELECT} * \\
        \textcolor{red}{FROM} users \\
        \textcolor{red}{WHERE} username = 'admin' \\
        \textcolor{red}{OR} '1'='1' \\
        \textcolor{red}{AND} password = 'anything'
        }
    \end{column}
\end{columns}

\end{frame}


\begin{frame}[fragile]{SQL Injection}

\begin{columns}[T] % T برای تراز بالای ستون‌ها
    \begin{column}{0.48\linewidth} % ستون سمت چپ
        \includegraphics[width=\linewidth]{3.3.png}
    \end{column}
    
    \begin{column}{0.48\linewidth} % ستون سمت راست
        \texttt{%
        \newline
        \newline
        \newline
        \textcolor{red}{SELECT} * \\
        \textcolor{red}{FROM} users \\
        \textcolor{red}{WHERE} username = 'admin'; \\
        \textcolor{red}{DROP TABLE} users; \\
        \textcolor{red}{AND} password = 'anything'
        }
    \end{column}
\end{columns}

\end{frame}

\begin{frame}{Prompt: Persona}
\begin{tcolorbox}
You are a senior application security engineer and database expert with hands-on experience finding, explaining, and remediating SQL Injection vulnerabilities across multiple languages and frameworks (PHP, Java, C#, Python, Node.js, Go, Ruby, etc.). Your tone is technical, precise and pragmatic. You reference OWASP best practices where relevant and always prefer safe, defensive guidance over exploit details.\
\end{tcolorbox}
\end{frame}

\begin{frame}{Prompt: Context}
\begin{tcolorbox}
[colback=blue!5!white,colframe=navy!75!black,title=Tasks]
You will be given one or more source files that interact with a database. The goal: determine whether SQL Injection exists, explain why (with exact code locations), provide a secure, runnable fix in the same language/framework, and produce tests that verify the fix. Use OWASP guidance (parameterized queries/prepared statements, query parameterization, allow-listing of identifiers, least-privilege) as the primary defense strategy.
\end{tcolorbox}
\end{frame}

\begin{frame}{Prompt: Tasks (Part 1)}
\begin{tcolorbox}[colback=blue!5!white,colframe=navy!75!black,title=Tasks]
\begin{enumerate}
  \item Analyze the code and identify any SQL Injection vulnerabilities. If none, explain why the code is safe.
  \item For each vulnerability found:
  \begin{itemize}
    \item Provide exact file/line references and an evidence snippet (<= 8 lines).
    \item Explain the technical root cause (e.g., dynamic string concatenation, unsafe ORM usage).
    \item Assign severity (LOW/MEDIUM/HIGH) with justification.
  \end{itemize}
  
\end{enumerate}
\end{tcolorbox}
\end{frame}

\begin{frame}{Prompt: Tasks (Part 2)}
\begin{tcolorbox}[colback=blue!5!white,colframe=navy!75!black,title=Tasks Continued]
  \begin{enumerate}
    \setcounter{enumi}{2} % continue numbering

    \item If identifiers are dynamic, show a secure allow-listing/mapping approach.
    \item Produce unit/integration tests that assert the fix prevents SQL injection (use DB mocks or in-memory DBs).
    \item Provide remediation plan, monitoring suggestions, and follow-up security checks (SAST, DAST, DB privileges review).
    \item NEVER output exploit payloads or instructions to exploit the vulnerability.
  \end{enumerate}
\end{tcolorbox}
\end{frame}

\begin{frame}{Result}
\begin{itemize}
\item Use parameterized queries (prepared statements) to separate code from data.
\item Employ stored procedures and input validation.
\item Apply the principle of least privilege on database accounts.
\item Regular security testing and updates to codebase.
\end{itemize}
\end{frame}

\begin{frame}{XSS}
\begin{itemize}
  \item Malicious scripts are permanently saved on the target server (e.g., in databases, comment sections, forums).
    \item Executes every time a user loads the affected content, impacting many users simultaneously.
    \item Represents the highest risk due to persistence and scale.

\end{itemize}
\end{frame}
\begin{frame}{Types of XSS}
\begin{itemize}
    \item Stored (persistent server storage of malicious scripts)
    \item Reflected (script comes from user request, reflected in server response)
    \item DOM-based (client-side modification of the Document Object Model)
\end{itemize}
\end{frame}


\section{References}
\begin{frame}{References}
\tiny
\begin{itemize}
\item O. Iyama, K. Kato, J. Miyachi, \textit{Derived categories of $N$-complexes}, 2017.
\item A. Neeman, \textit{The homotopy category of flat modules and Grothendieck duality}, 2008.
\item OWASP Foundation, \textit{SQL Injection Prevention Cheat Sheet}.
\item Vaswani et al., \textit{Attention is All You Need}, 2017.
\item Radford et al., \textit{Language Models are Few-Shot Learners}, OpenAI, 2020.
\end{itemize}
\end{frame}

\begin{frame}
\vspace{3cm}
\begin{center}
\begin{LARGE}
\textcolor{blue}{\textbf{Thank you all for your attention}}
\end{LARGE}
\end{center}
\end{frame}

\end{document}
