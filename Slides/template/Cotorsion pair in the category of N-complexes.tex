\documentclass[t,ignorenonframetext]{beamer}
%\usepackage{beamerthemeJuanLesPins}%
%\usepackage{beamercolorthemecougar}
\usepackage{beamerinnerthemecircles}
\usepackage[all]{xy}
\mode<presentation>
{
\usetheme{Warsaw}
\usecolortheme{dolphin}
\setbeamercolor{frametitle}{fg=black,bg=lightgray}
\usefonttheme{structurebold}
}


% The line below is what I talked about that makes all
% items in a list into overlays
%\beamerdefaultoverlayspecification{<+->}

%\newcommand{\tc}[1]{$\backslash$\texttt{#1}}


\newcommand{\bb}{{\rm b}}

\newcommand{\rt}{\rightarrow}
\newcommand{\lrt}{\longrightarrow}
\newcommand{\ve}{\delta}
\newcommand{\st}{\stackrel}
\newcommand{\pa}{\partial}
\newcommand{\al}{\alpha}
\newcommand{\La}{\Lambda}
\newcommand{\la}{\lambda}
\newcommand{\Ga}{\Gamma}
\newcommand{\ga}{\gamma}

\newcommand{\tc}{\otimes_{\CC}}

\newcommand{\lan}{\langle}
\newcommand{\ran}{\rangle}

\newcommand{\CA}{\mathcal{A} }
\newcommand{\CB}{\mathcal{B}}
\newcommand{\Ab}{\mathcal{A}\mathfrak{b} }
\newcommand{\CC}{\mathcal{C} }
\newcommand{\DD}{\mathcal{D} }
\newcommand{\CE}{\mathcal{E}}
\newcommand{\CF}{\mathcal{F} }
\newcommand{\CG}{\mathcal{G} }
\newcommand{\GG}{\mathcal{GG} }
\newcommand{\GP}{\mathcal{GP} }
\newcommand{\GI}{\mathcal{GI} }
\newcommand{\CI}{\mathcal{I} }
\newcommand{\CJ}{\mathcal{J} }
\newcommand{\CK}{\mathcal{K} }
\newcommand{\CM}{\mathcal{M} }
\newcommand{\CN}{\mathcal{N} }
\newcommand{\CP}{\mathcal{P} }
\newcommand{\CQ}{\mathcal{Q} }
\newcommand{\CR}{\mathcal{R} }
\newcommand{\CS}{\mathcal{S} }
\newcommand{\CT}{\mathcal{T} }
\newcommand{\CX}{\mathcal{X} }
\newcommand{\CXb}{\mathcal{X}^{\bullet}}
\newcommand{\CY}{\mathcal{Y} }
\newcommand{\CV}{\mathcal{V}}
\newcommand{\CU}{\mathcal{U}}

\newcommand{\BF}{\mathbf{F}}
\newcommand{\BP}{\mathbf{P}}
\newcommand{\BI}{\mathbf{I}}
\newcommand{\BJ}{\mathbf{J}}
\newcommand{\BZ}{\mathbf{Z}}
\newcommand{\CZ}{\mathcal{Z} }
\newcommand{\CCD}{\mathcal{D} }
\newcommand{\Sf}{\mathcal{S}f}
\newcommand{\Lf}{\mathcal{L}f}
\newcommand{\CW}{\mathcal{W}}

\newcommand{\C}{\mathbb{C} }
\newcommand{\D}{\mathbb{D} }
\newcommand{\K}{\mathbb{K} }
\newcommand{\N}{\mathbb{N} }
\newcommand{\R}{\mathbb{R} }
\newcommand{\Z}{\mathbb{Z} }
\newcommand{\M}{\mathbf{M}}
\newcommand{\A}{\mathbf{A}}
\newcommand{\B}{\mathbf{B}}
\newcommand{\W}{\mathbf{W}}
\newcommand{\X}{\mathbf{X}}
\newcommand{\Y}{\mathbf{Y}}
\newcommand{\BG}{\mathbf{G}}
\newcommand{\PP}{\mathbf{P}}
\newcommand{\BD}{\mathbf{D}}
\newcommand{\DQ}{\mathbb{D}(\CQ)}
\newcommand{\TT}{\mathbb{T} }
\newcommand{\Tt}{\mathbf{T} }
\newcommand{\HE}{{\rm{H}}}
\newcommand{\BE}{{\rm{B}}}
\newcommand{\CKK}{{\rm{C}}}


\newcommand{\Mod}{{\rm{Mod}} \mbox{-}}
\newcommand{\mmod}{{\rm{mod}} \mbox{-}}
\newcommand{\Sum}{{\rm Sum} \mbox{-}}
\newcommand{\summ}{{\rm sum} \mbox{-}}
\newcommand{\Prj}{{\rm Prj} \mbox{-}}
\newcommand{\prj}{{\rm prj} \mbox{-}}
\newcommand{\Inj}{{\rm Inj} \mbox{-}}
\newcommand{\inj}{{\rm inj} \mbox{-}}
\newcommand{\Injop}{{\rm Inj}^{{\rm op}} \mbox{-}}
\newcommand{\GPrj}{{\rm GPrj} \mbox{-}}
\newcommand{\Gprj}{{\Gp\mbox{-}}}
\newcommand{\GInj}{{\rm GInj} \mbox{-}}
\newcommand{\Ginj}{{\Gi \mbox{-}}}
\newcommand{\Flat}{{\rm Flat} \mbox{-}}
\newcommand{\gr}{{\rm gr}\mbox{-}}
\newcommand{\op}{{\rm{op}}}
\newcommand{\ac}{{\rm{ac}}}
\newcommand{\Rmod}{{\rm{mod}} \text{-} R}
\newcommand{\rep}{{\rm rep}}
\newcommand{\Rep}{{\rm Rep}}
%\newcommand{\DG}{{\rm DG}}
\newcommand{\om}{\omega}
\newcommand{\DDGPrj}{{\rm{DGPrj}}}
\newcommand{\ddgPrj}{{\rm{DGPrj}}}
\newcommand{\ddgInj}{{\rm{DGInj}}}
\newcommand{\dgPrj}{{\rm{DGPrj}\mbox{-}}}
\newcommand{\dgInj}{{\rm{DGInj}\mbox{-}}}
\newcommand{\DG}{{\rm dg\mbox{-}}}


\newcommand{\ex}{{\rm{ex}}}
\newcommand{\cw}{{\rm{cw}}}

\newcommand{\Cok}{{\rm{Cok}}}


\newcommand{\KPQ}{{\K({\Prj} \ \CQ)}}
\newcommand{\KIQ}{{\K({\Inj} \ \CQ)}}

\newcommand{\RGInj}{{\rm{GInj}} \ R}
\newcommand{\RGPrj}{{\rm{GPrj}} \ R}
\newcommand{\QR}{{{\rm Rep}{(\CQ, R)}}}

\newcommand{\Tr}{{\rm Tr}}
\newcommand{\im}{{\rm{Im}}}
%\newcommand{\op}{{\rm{op}}}
\newcommand{\inc}{{\rm{inc}}}
\newcommand{\can}{{\rm{can}}}
\newcommand{\cone}{{\rm{Cone}}}
\newcommand{\vcone}{{\rm{VCone}}}
\newcommand{\TS}{{\rm{S}}}
\newcommand{\VS}{{\rm{VS}}}
\newcommand{\pd}{{\rm{pd}}}
\newcommand{\id}{{\rm{id}}}
\newcommand{\fd}{{\rm{fd}}}
\newcommand{\Gid}{{\rm{Gid}}}
\newcommand{\Gpd}{{\rm{Gpd}}}
\newcommand{\Gfd}{{\rm{Gfd}}}
\newcommand{\gldim}{{\rm{gl.dim}}}
\newcommand{\HT}{{\rm{H}}}
\newcommand{\TE}{{\rm{E}}}
\newcommand{\Coker}{{\rm{Coker}}}
\newcommand{\Ker}{{\rm{Ker}}}
\newcommand{\Add}{{\rm{Add}\mbox{-}}}
\newcommand{\add}{{\rm add} \mbox{-}}
\newcommand{\Syz}{{\rm{Syz}}}
\newcommand{\Ann}{{\rm{Ann}}}
\newcommand{\Spec}{{\rm{Spec}}}
\newcommand{\grade}{{\rm{grade}}}
\newcommand{\depth}{{\rm{depth}}}
\newcommand{\Tor}{{\rm{Tor}}}
\newcommand{\Hom}{{\rm{Hom}}}
\newcommand{\End}{{\rm End}}
\newcommand{\CHom}{\mathcal{H} om}
\newcommand{\Ext}{{\rm{Ext}}}
\newcommand{\ext}{{\rm{ext}}}
\newcommand{\Hext}{{\rm{\widehat{ext}}}}
\newcommand{\HExt}{{\rm{\widehat{Ext}}}}
\newcommand{\Text}{{\rm{\widetilde{ext}}}}
\newcommand{\TExt}{{\rm{\widetilde{Ext}}}}
\newcommand{\LExt}{{\rm{\overline{Ext}}}}
\newcommand{\Loc}{{\rm Loc}}
\newcommand{\Vecc}{{\rm Vec}}
\newcommand{\vecc}{{\rm vec}}
\newcommand{\Gp}{{\mathcal{G}p}}
\newcommand{\Gi}{{\mathcal{G}i}}
\newcommand{\Filt}{{\rm{Filt}}}
\newcommand{\ZE}{{\rm{Z}}}
\newcommand{\dg}{{\rm{dg}}}

\newcommand{\Prjj}{{\rm{Prj}}}
\newcommand{\prjj}{{\rm{prj}}}
\newcommand{\Injj}{{\rm{Inj}}}


\newcommand{\Fmod}{\rm mod}
\newcommand{\FMod}{\rm Mod}
\newcommand{\Mor}{{\rm{Mor}}}

\newcommand{\KIR}{{\K({\Inj} \ R)}}
\newcommand{\KPR}{{\K({\Prj} \ R)}}
\newcommand{\KFR}{{\K({\Flat} \ R)}}


\newcommand{\limt}{\underset{\underset{\alpha < \beta}{\longrightarrow}}{\lim}}



\newcommand{\II}{\textbf{I}}
\newcommand{\QQ}{\textbf{Q}}

\newcommand{\BC}{\mathbb{C}}
\newcommand{\Ho}{{\rm{Ho}}}


\title[Cotorsion pairs in the category of $N$-complexes]{Cotorsion pairs in the category of $N$-complexes}
\author[P. Bahiraei]{Payam Bahiraei}
\institute[IPM]{\\~ \\ University of Guilan and IPM\\~
 \\
}

\date[Jan. 15, 2020]{January 15, 2020}
\begin{document}
\frame{
\maketitle
}
%\frame{\frametitle{Overview}
%\tableofcontents
%}


\section{Homotopy category of $N$-complexes}
\begin{frame}{\textcolor{magenta}{1) $N$-complexes:}}

Let $\CA$ be an abelian category and fix a positive integer $N\geq 2$.

\setbeamercovered{transparent}

\pause
An $N$-complex $X=(X^i,d^i_X)$ is a diagram
\[ \xymatrix { \cdots \ar[r] & X^{n-1} \ar[r]^{d^{n-1}_X} & X^n \ar[r]^{d^{n}_X} & X^{n+1} \ar[r]^{d^{n+1}_X} & \cdots }\]

\pause
With $X^i\in \CA$ and $d^{i+N-1}_X \cdots d^{i+1}_X d^i_X=0$ for any $i\in \Z$.

\vspace{0.4cm}
\pause
For $0\leq r<N$ and $i\in \mathbb{Z}$, we often denote 
$$ d_{X,\lbrace r\rbrace}^{i}:=d_{X}^{i+r-1} \cdots d_{X}^{i} $$

\pause
In this notation $d_{X,\lbrace 1\rbrace}^i=d^i_{X}$ and $d_{X,\lbrace 0\rbrace}^i=1_{X_i}$.

\end{frame}



\begin{frame}
A morphism $f:X\lrt Y$ between $N$-complexes is a commutative diagram
\vspace{0.4cm}
\pause
\[ \xymatrix { \cdots \ar[r] & X^{n-1} \ar[r]^{d^{n-1}} \ar[d]^{f^{n-1}}& X^n \ar[r]^{d^{n}}\ar[d]^{f^{n}} & X^{n+1} \ar[r]^{d^{n+1}} \ar[d]^{f^{n+1}}& \cdots
\\ \cdots \ar[r] & Y^{n-1} \ar[r]^{e^{n-1}} & Y^n \ar[r]^{e^{n}} & Y^{n+1} \ar[r]^{e^{n+1}} & \cdots }\]

\pause
We denote by $\C_N(\CA)$ the category of unbounded $N$-complexes over $\CA$.
\end{frame}

\begin{frame}
\setbeamercovered{transparent}
Let $\mathcal{S}_N(\CA)$ be the collection of short exact sequence in $\C_N(\CA)$ of which each term is split exact.
 
\vspace{0.4cm}
\pause
\begin{itemize}[<+->]
\item $(\C_N(\CA),\mathcal{S}_N(\CA))$ is an exact category.
\item  $(\C_N(\CA),\mathcal{S}_N(\CA))$ is a Frobenius category.
\end{itemize}

\pause
Indeed,

\pause
For any object $M$ of $\CA$, $j\in \Z$ and $1\leq i \leq N$, let
\begin{small}
$$\xymatrixcolsep{2.20pc}\xymatrix{  D^{j}_{i}(M): \cdots \ar[r] & 0 \ar[r] & X^{j-i+1} \ar[r]^{\,\,\,\,{d}^{j-i+1}_X} & \cdots \ar[r]^{{d}^{j-2}_X} & X^{j-1} \ar[r]^{{d}^{j-1}_X}  & X^j \ar[r] &0}$$
\end{small}
be an $N$-complex satisfying $X^n=M$ for all $j-i+1\leq n \leq j$ and $d^{n}_X=1_M$ for all $j-i+1\leq n \leq j-1$.
\end{frame}

\begin{frame}
\setbeamercovered{transparent}
\begin{itemize}[<+->]
\item $D_{N}^{i}(M)$ is an $\mathcal{S}_N$-projective and $\mathcal{S}_N$-injective object of $(\C_N(\CA),\mathcal{S}_N(\CA))$.
\item  Let $X\in \C_N(\CA)$ be given. Then we have the following exact sequences in $\mathcal{S}_N(\CA)$
\end{itemize}
\pause
\[
\xymatrix{0 \ar[r] & \Ker \rho_X \ar[r]^{\varepsilon_X \qquad\quad} & \bigoplus_{n\in \Z} D_{N}^{n}(X^{n-N+1}) \ar[r]^{\qquad \quad \,\,\,\,\,\, \rho_X} & X \ar[r]  & 0 }
\]
\pause 
and
\[
\xymatrix{0 \ar[r] & X \ar[r]^{\lambda_X \qquad\quad} & \bigoplus_{n\in \Z} D_{N}^{n}(X^{n}) \ar[r]^{\quad  \eta_X} & \Cok \lambda_X  \ar[r]  & 0 }
\]
\pause
\begin{itemize}[<+->]
\item  $(\C_N(\CA),\mathcal{S}_N(\CA))$ has enough projectives and enough injectives.
\end{itemize}
 
\end{frame}

\begin{frame}{The homotopy category of $N$-complexes:}
\setbeamercovered{transparent}
\pause
A morphism $f:X \longrightarrow Y$ of $N$-complexes is called \textcolor{red}{null-homotopic} if there exists $s^i \in \Hom_{\CA}(X^i,Y^{i-N+1})$ such that
\pause
\[ f^i=\sum_{j=0}^{N-1} d_{Y, \lbrace N-1-j\rbrace}^{i-(N-1-j)}s^{i+j}d^{i}_{X, \lbrace j\rbrace}\]
\pause
For morphisms $f,g:X\lrt Y$ in $\C_N(\CA)$, we denote $f\sim g$ if $f-g$ is null-homotopic.
\pause

\vspace{0.3cm}
$\K_N(\CA)$ : The homotopy category of unbounded $N$-complexes.
\begin{itemize}
\item $Obj(\K_N(\CA))=Obj(\C_N(\CA))$
\item $\Hom_{\K_N(\CA)}(X,Y)=\Hom_{\C_N(\CA)}(X,Y)/\sim$
\end{itemize}

\end{frame}

\begin{frame}
\setbeamercovered{transparent}
\begin{block}{Theorem [IKM]}
The stable category of the frobenius category $(\C_N(\CA),\mathcal{S}_N(\CA))$
is the homotopy category $\K_N(\CA)$ of $\CA$.
\end{block}
\pause
Therefore, 
\pause
%\begin{center}
%$f:X\lrt Y$ is null-homotopic $\Leftrightarrow$ f factors through the morphism $\lambda_X:X\lrt \bigoplus_{n\in \Z} D_{N}^{n}(X^{n})$
%\end{center}
%\pause

\vspace{0.3cm}
\begin{itemize}
\item $\K_N(\CA)$ is a triangulated category.
\end{itemize}
\pause

\vspace{0.3cm}
\textcolor{blue}{What is the suspension functor of $\K_N(\CA)$?}
\end{frame}


\begin{frame}
\setbeamercovered{transparent}
Define functors $\Sigma,\Sigma^{-1}:\C_N(\CA)\lrt \C_N(\CA)$ by
$$ \Sigma^{-1} X=\Ker \rho_X \qquad \,\,\,\, \text{and} \qquad \,\,\,\, \Sigma X=\Cok \lambda_X$$
\pause
\begin{tiny}
$$(\Sigma X)^m={\bigoplus}_{i=m+1}^{m+N-1}X^i, \qquad d^{m}_{\Sigma X}=
\left[ \begin{array}{cccccc}


0 & 1 & 0 & 0  &\cdots & 0 \\
 & 0 & \ddots & \ddots & \ddots & \vdots \\
  \vdots & \vdots & \ddots & \ddots & \ddots & 0 \\
   &  &  & \ddots & \ddots & 0 \\
    0 & 0 & \cdots & \cdots & 0 & 1 \\
     -d_{\lbrace N-1\rbrace}^{m+1} & -d_{\lbrace N-2\rbrace}^{m+2} & \cdots & \cdots & \cdot & -d^{m+N-1} \\
\end{array} \right]     
$$
$$(\Sigma^{-1} X)^m={\coprod}^{i=m-1}_{m-N+1}X^i, \qquad d^{m}_{\Sigma^{-1} X}=
\left[ \begin{array}{cccccc}


-d^{m-1} & 1 & 0 & \cdots  &\cdots & 0 \\
-d_{\lbrace 2\rbrace}^{m-1} & 0 & 1 & \ddots & \ddots & \vdots \\
  \vdots & \vdots & \ddots & \ddots & \ddots & 0 \\
   &  &  & \ddots & \ddots & 0 \\
    -d_{\lbrace N-2\rbrace}^{m-1} & 0 & \cdots & \cdots & 0 & 1 \\
     -d_{\lbrace N-1\rbrace}^{m-1} & 0 & \cdots & \cdots & \cdot & 0 \\
\end{array} \right]     
$$
\end{tiny}
\end{frame}

\section{Derived category of $N$-complexes}

\begin{frame}{\textcolor{magenta}{2)Derived category of $N$-complexes}}
\setbeamercovered{transparent}
Let $X$ be an $N$-complex in $\CA$
\pause

For $0\leq r \leq N$ and $i\in \Z$ define
$$\ZE^{i}_{r}(X):= \Ker d^{i}_{X,\lbrace r \rbrace} \quad , \quad \BE^{i}_{r}(X):= \im d^{i-r}_{X, \lbrace r \rbrace}
$$
$$
  \quad \HE^{i}_{r}(X):= \ZE^{i}_{r}(X)/ \BE^{i}_{N-r}(X).$$
\pause
 
For example, $\ZE^{n}_{N}(X)=\BE^{n}_{0}(X)=X^n$  and $\ZE^{n}_{0}(X)=\BE^{n}_{N}(X)=0$

\vspace{0.3cm}
\pause
we can understand a homology as follows
\begin{small}
$$\HE^{n}_{r}(X)=\Cok(\ZE^{n-N+r}_{N}(X)\xrightarrow{d^{n-N+r}_N}\cdots \xrightarrow{d^{n-2}_{r+2}} \ZE^{n-1}_{r+1}(X)\xrightarrow{d^{n-1}_{r+1}} \ZE^{n}_{r}(X))
$$
\end{small}

  

\end{frame}




\begin{frame}
\setbeamercovered{transparent}
\begin{block}{Definition:}
Let $X \in \K_N(\CA)$. We say $X$ is \textcolor{red}{$N$-exact} if $\HE^{i}_{r}(X)=0$
for each $i \in \mathbb{Z}$ and all $r=1,2,...,N-1$. We denote the full subcategory of $\K_N(\CA)$ consisting of $N$-exact complexes by $\K_{N}^{\ac}(\CA)$ . 
\end{block}

%\pause
%\begin{block}{Proposition [IKM]}
%$\K_{N}^{\ac}(\CA)$ is a thick subcategory of $\K_{N}(\CA)$.
%\end{block}

\pause
\begin{block}{Definition:}
A morphism $f:X\lrt Y$ of $\K_N(\CA)$ is called an \textcolor{red}{$N$-quasi-isomorphism} if $\HE^{i}_{r}(f):\HE^{i}_{r}(X)\lrt \HE^{i}_{r}(Y)$ is an isomorphism for any $0<r<N$ and $i\in \Z$.
\end{block}

\end{frame}



\begin{frame}
\setbeamercovered{transparent}

\vspace{0.6cm}
\pause
\begin{itemize}
\item The derived category of $N$-complexes is defined as the quotient category
$$\D_N(\CA)=\K_N(\CA)/\K_N^{\ac}(\CA) $$
\end{itemize}

\vspace{0.3cm}
\pause
By definition, a morphism in $\K_N(\CA)$ is an $N$-quasi-isomorphism iff it is an isomorphism in $\D_N(\CA)$.

\vspace{0.3cm}

\end{frame}

\section{Cotorsion Pairs}
\begin{frame}{\textcolor{magenta}{3) Cotorsion Pairs}}
\setbeamercovered{transparent}
\only<1>{
\begin{block}{Definition}
A pair $(\CF,\CC)$ of classes of object of $\CA$ is said to be a \textcolor{red}{cotorsion pair}
if $\mathcal{F}^\perp=\mathcal{C}$ and $\mathcal{F} = {}^\perp\mathcal{C}$, where the left and right orthogonals are defined as follows 
\[{}^\perp\mathcal{C}:=\{A \in \CA\ | \ \Ext^1_{\CA}(A,Y)=0, \ {\rm{for \ all}} \ Y \in \mathcal{C} \}\]
and
\[\mathcal{F}^\perp:=\{A \in \CA\ | \ \Ext^1_{\CA}(W,A)=0, \ {\rm{for \ all}} \ W \in \mathcal{F} \}.\] 
\end{block}}
\pause
\only<2->{
\begin{block}{}
A cotorsion pair  $(\mathcal{F}, \mathcal{C})$ is called \textcolor{red}{complete} if for every $A \in \CA$ there exist exact sequences
\[0 \rt Y \rt W \rt A \rt 0 \ \ \ {\rm and} \ \ \ 0 \rt A  \rt Y' \rt W' \rt 0,\]
where $W, W'\in \mathcal{F}$ and $Y, Y' \in  \mathcal{C}$.
\end{block}}
\pause
\only<3->{
\begin{block}{}
If there is a set $\mathcal{S}$ of objects of $\CA$ such that $\mathcal{S}^\perp=\mathcal{C}$ for some cotorsion pair
$(\mathcal{F}, \mathcal{C})$, then the pair is said to be \textcolor{red}{cogenerated by a set}.
\end{block}}

\end{frame}

\section{Cotorsion pair in $\C_N(\CA)$}

\begin{frame}{\textcolor{magenta}{4) Cotorsion pair in $\C_N(\CA)$}}
\setbeamercovered{transparent}
\begin{block}{Definition:}
Let $\CA$ be an abelian category. Given two classes of objects $\CX$ and $\CU$ in
$\CA$ with $\CU\subseteq \CX$. We denote by $\widetilde{\CU}_{\CX_N}$ the class of all $N$-exact complexes $U$ with each degree
$U^i\in \CU$ and each cycle $\ZE_{r}^{i}(U) \in \CX$ for all $1\leq r\leq N-1$ and $i\in \Z$.
\end{block}

\pause
\only<2>{
\begin{block}{Proposition:}
Let $\CA$ be an abelian category with injective cogenerator $J$. Let $(\CU,\CV)$ and $(\CX,\CY)$ be two cotorsion pairs with $\CU\subseteq \CX$ in $\CA$. Then
$( \widetilde{\CU}_{\CX_N}, (\widetilde{\CU}_{\CX_N})^\perp)$ is a cotorsion pair in $\C_N(\CA)$ and $(\widetilde{\CU}_{\CX_N})^\perp$ is the class of all $N$-complexes $V$
for which each $V^i \in \CV$ and for each map $U\rightarrow V$ is null-homotopic whenever $U\in\widetilde{\CU}_{\CX_N}$.
\end{block}}

\pause
\only<3>{
\begin{block}{Proposition:}
Let $(\CU,\CV)$ and $(\CX,\CY)$ be two cotorsion pairs with $\CU\subseteq \CX$ in $\CA$. Then
$( {}^\perp(\widetilde{\CY}_{\CV_N}), \widetilde{\CY}_{\CV_N})$ is a cotorsion pair in $\C_N(\CA)$ and ${}^\perp(\widetilde{\CY}_{\CV_N})$ is the class of all $N$-complexes $X$
for which each $X^i \in \CX$ and for each map $X\rightarrow Y$ is null-homotopic whenever $Y\in\widetilde{\CY}_{\CV_N}$.
\end{block}}
\end{frame}
\begin{frame}
\setbeamercovered{transparent}
Let $(\CF,\CC)$ be a cotorsion pair in $\CA$ and consider the following subclasses of $\C_N(\CA)$
\pause

\begin{itemize}
\begin{small}
\item[(1)]$C_N(\CF)=\lbrace X \in \C_N(\CA)\, | \, X^i \in \CF\rbrace $ 
\pause
\item[(2)] $C_N(\CC)=\lbrace X \in \C_N(\CA)\, | \, X^i \in \CC\rbrace $
\pause 
\item[(3)] $\widetilde{\mathcal{F}}_N=\lbrace X \in \mathcal{E}_N \, | \, \ZE_{r}^{i}(X) \in \mathcal{F}\rbrace$ 
\pause
\item[(4)] $\widetilde{\mathcal{C}}_N=\lbrace Y \in \mathcal{E}_N \, | \, \ZE_{r}^{i}(Y) \in \mathcal{C}\rbrace$ 
\pause
\item[(5)]$\dg \widetilde{\mathcal{F}}_N=\lbrace X \in \C_N(\mathcal{F})\, | \, \Hom_{\K_{N}(\CA)}(X,C)=0 \, \,  , \, \forall C\in \widetilde{\mathcal{C}}_N \rbrace$  
\pause
\item[(6)] $\dg \widetilde{\mathcal{C}}_N=\lbrace X \in \C_N(\mathcal{F})\, | \, \Hom_{\K_{N}(\CA)}(X,C)=0 \, \, , \, \forall C\in \widetilde{\mathcal{C}}_N \rbrace$ 
\pause
\item[(7)]$\text{ex}\widetilde{\mathcal{F}}_N=\C_N(\mathcal{F})\cap \mathcal{E}_N$
\pause
\item[(8)]$\text{ex}\widetilde{\mathcal{C}}_N=\C_N(\mathcal{C})\cap \mathcal{E}_N$

\end{small}
\end{itemize} 
\end{frame}
\begin{frame}
\setbeamercovered{transparent}
\begin{itemize}
\item[$\blacktriangleright$] If we consider $(\CU,\CV)=(\CF,\CC)$ then clearly $\widetilde{\CF}_{\CF_N}=\widetilde{\CF}_N$ and $(\widetilde{\CF}_{\CF_N})^\perp=dg\widetilde{\CC}_N$, so $(\widetilde{\CF}_N,dg\widetilde{\CC}_N)$ is a cotorsion pairs.

\pause
\item[$\blacktriangleright$]Dually $(dg\widetilde{\CF}_N,\widetilde{\CC}_N)$ is a cotorsion pair.
\pause

\vspace{0.3cm}
\item[$\blacktriangleright$]If we set $(\CU,\CV)=(\CA,\CI)$ (the usual injective cotorsion pair in $\CA$), then $\CF\subseteq \CA$, therefore $\widetilde{\CF}_{\CA_N}=\text{ex}\widetilde{\CF}_N$, so $(\text{ex}\widetilde{\CF}_N,(\text{ex}\widetilde{\CF}_N)^\perp)$ is a cotorsion pair.
\pause

\vspace{0.3cm}
\item[$\blacktriangleright$] Dually $({}^\perp(\text{ex}\widetilde{\CC}_N),\text{ex}\widetilde{\CC}_N)$ is a cotorsion pair.
\end{itemize}

\vspace{0.1cm}
\begin{block}{Proposition:}
Let $(\CF,\CC)$ be a cotorsion pair in $\CA$. Then $(\C_N(\CF), {\C_N(\CF)}^\perp)$ and $({}^\perp{\C_N(\CC)}, \C_N(\CC))$ are cotorsion pairs in $\C_N(\CA)$.
\end{block} 
\end{frame}
\begin{frame}{Main Theorem:}
\setbeamercovered{transparent}
\only<1>{
Let $\CG$ be a Grothendieck category endowed with a faithful functor $U:\CG\rightarrow\textbf{Set}$, 

We
will also assume that there exists an infinite regular cardinal $\lambda$ such that for each $G\in \CG$ and
any set $S\subseteq G$ with $|S|<\lambda$, there is a subobject $X\subseteq G$ such that $S \subseteq X \subseteq G$ and $|X|<\lambda$.
\pause
}
\begin{block}{Theorem:}
Let $(\CF,\CC)$  be a cotorsion pair in $\CG$ cogenerated by a set such that $\CF$ contains a generator $G$ of $\CG$. Then the induced pairs
\begin{itemize}
\item[(1)]$(\widetilde{\CF}_N,dg\widetilde{\CC}_N)$ and $(dg\widetilde{\CF}_N,\widetilde{\CC}_N)$
\item[(2)]$(\C_N(\CF), {\C_N(\CF)}^\perp)$ and $({}^\perp{\C_N(\CC)}, \C_N(\CC))$
\item[(3)]$(ex\widetilde{\CF}_N,(ex\widetilde{\CF}_N)^\perp)$ and $({}^\perp(ex\widetilde{\CC}_N),ex\widetilde{\CC}_N)$
\end{itemize}
are complete cotorsion pairs.

\end{block}

\end{frame}

\section{Applications}

\begin{frame}{\textcolor{magenta}{5) Applications }}
\setbeamercovered{transparent}
\begin{block}{Proposition:}
If $(\CF,\CC)$ is a complete cotorsion pair in $\C_N(\CA)$ and if $\CF$ is closed under taking suspensions, then the embeddings $\K_N(\CF)\rightarrow \K_N(\CA)$ and $\K_N(\CC)\rightarrow \K_N(\CA)$ have right and left adjoints respectively.
\end{block}
\pause

\vspace{0.3cm}
\begin{itemize}
\item Consider the complete cotorsion pair $(\Prj R, \Mod R)$.
\pause
\item Then we have $(\C_N(\Prj R), \C_N(\Prj R)^\perp)$ in $\C_N(R)$.
\pause
\item $\K_N(\Prj R)\rightarrow \K_N(\Mod R)$ has right adjoint functor $j:\K_N(\Mod R)\rightarrow \K_N(\Prj R)$.
\pause
\item Then the natural inclusion $j_{!}:\K_N(\Prj R)\rightarrow \K_N(\Flat R)$ has a right adjoint $j^\ast=j|_{\K(\Flat R)}$. ($N$-complex version of Neeman result \cite[Proposition 8.1]{Neeman1})
\end{itemize}
 



\end{frame}

\begin{frame}
\setbeamercovered{transparent}
By general theory on Bousfield localization we can say that the right adjoint functor $j^\ast:\K_N(\Flat R)\rightarrow \K_N(\Prj R)$ is a Verdier quotient. 

\pause
In fact this adjoint functor identifies $\K_N(\Prj R)$ with the Verdier quotient map
$$\K_N(\Prj R)\rightarrow \K_N(\Flat R)/\K_N(\Prj R)^\perp$$

\pause
where \begin{small}
$$\K_N(\Prj R)^\perp=\lbrace Y\in \K_N(\Flat R)\, | \, \Hom(j_{!}X,Y)=0\,:\, \forall X\in \K_N(\Prj R)\rbrace$$
\end{small}
 
\pause

\vspace{0.3cm}
$\K_N(\Prj R)^\perp$ concides with $\K_N(\widetilde{\Flat R})$ ( definition of $\widetilde{\CF}_N$)
\end{frame}

\begin{frame}
\setbeamercovered{transparent}
But $(\widetilde{\Flat R}_N,\widetilde{\Flat R}^\perp_N)$ is complete cotorsion pair
\pause

\vspace{0.2cm}
So $\K_N(\Prj R)^\perp=\K_N(\widetilde{\Flat R})\rightarrow \K_N(\Flat R)$ admits a right adjoint functor.
\pause

\vspace{0.2cm}
On the other hand,
$$\K_N(\Prj R)^\perp\rightarrow \K_N(\Flat R)\xrightarrow{j^\ast} \K_N(\Prj R)$$
is quotient sequence, hence it is a localization sequence.

\pause
\vspace{0.2cm}
Therefore $ \K_N(\Flat R)\xrightarrow{j^\ast} \K_N(\Prj R)$ has a right adjoint functor.($N$-complex version of Neeman result \cite[Theorem 0.1]{Neeman2})
\end{frame}


\section{References}
\begin{frame}
\begin{thebibliography}{9999}
\tiny

\bibitem [B]{B} {\sc P. Bahiraei,} {\sl Cotorsion pairs and adjoint functors in the homotopy category of N-complexes , } J. Algebra Appl (2019),(Accepted)

\bibitem {BEIJR} {\sc D. Bravo, E.E. Enochs, A.C. Iacob, O.M.G. Jenda, J. Rada,} {\sl Cotorsion pairs in ${\bf C}(R \mbox{-} {\rm Mod})$,} Rocky Mountain J. Math. {\bf 42} (2012), 1787-1802.

\bibitem [Gil04]{Gil04} {\sc J. Gillespie,} {\sl The flat model structure on $Ch(R)$, } Trans. Amer. Math. Soc. {\bf 356} (2004), no. 8, 3369-3390.

\bibitem [Gil08]{Gil08} {\sc J. Gillespie,} {\sl Cotorsion pairs and degreewise homological model
structures, } Homol Homotopy Appl. {\bf 10} (2008), no. 1, 283-304.

\bibitem [IKM14]{IKM14} {\sc O. Iyama, K. Kato, and J. Miyachi,} {\sl Derived categories of $N$-complexes, }available at arXiv:1309.6039v5, 2017.

\bibitem [IKM16]{IKM16} {\sc O. Iyama, K. Kato, and J. Miyachi,} {\sl Polygon of recollement and $N$-complexes, }available at arXiv:1603.06056, 2016.

\bibitem [Mur]{Mur} {\sc D. Murfet,} {\sl The Mock homotopy category of 
projectives and Grothendieck duality, } PhD thesis, Aust. National U. 2008.

\bibitem [Neeman1]{Neeman1} {\sc A. Neeman,} {\sl The homotopy category of 
at modules, and grothendieck duality, }Inv. 
mathematicae, {\bf 174} (2008) 255-308.

\bibitem [Neeman2]{Neeman2} {\sc A. Neeman,} {\sl Some adjoints in homotopy category, }Ann. 
Math, {\bf 171} (2010) 2143-2155.

\bibitem [St]{St} {\sc J. \v{S}\v{t}ov\'{i}\v{c}ek,} {\sl Deconstructibility and Hill lemma in Grothendieck categories,} Forum Math. {\bf 25} (2013),  193-219.

\end{thebibliography}
\end{frame}


\begin{frame}
\vspace{3cm}
\begin{center}
\begin{LARGE}
\textcolor{blue}{\textbf{Thank you all for your attention}}
\end{LARGE}
\end{center}

\end{frame}

\end{document}
